\documentclass[12pt]{article}

\usepackage{neurips_2021}

\usepackage[utf8]{inputenc} % allow utf-8 input
\usepackage[T1]{fontenc}    % use 8-bit T1 fonts
\usepackage{hyperref}       % hyperlinks
\usepackage{url}            % simple URL typesetting
\usepackage{booktabs}       % professional-quality tables
\usepackage{amsfonts}       % blackboard math symbols
\usepackage{nicefrac}       % compact symbols for 1/2, etc.
\usepackage{microtype}      % microtypography
\usepackage{xcolor}

\newcommand{\contentdescription}[1]{}

\title{Report}

\author{
    Pavlo Melnyk,
    Julian Alfredo Mendez,
    Emanuel S\'{a}nchez Aimar
}

\begin{document}

    \maketitle

    \begin{abstract}
        \contentdescription{
            Abstract (5-10\%) :
            Give an overview of what you have done in the project with the key results and findings of your work.
            Should be no more than 300 words.
        }

    \end{abstract}


    \section{Introduction}

    \contentdescription{
        Introduction (5-15\%):
        Describe the problem, the approach of the paper, the experiments, and the results.
        At the high-level talk about what you worked on in your project and why it is important.
        Then give an overview of your results.
    }


    \section{Related Work}
    \contentdescription{
        Related Work (5-15\%): Discuss the published work related to your project paper, the types of experiments you do and the additional method that you have added to this work or you have compared this paper with (if any).
    }


    \section{Methods}
    \contentdescription{
        Methods (15-25\%): Describe the original paper's method to the extent that you would need to make your report and findings understandable. Otherwise, here you can describe other methods that you compare with or other methods that you apply on top of what you reimplemented. Here, you also try to justify any methodical modification or incremental changes that you have added to the original paper. It may be helpful to include figures, diagrams, or tables to describe your method or compare it with other methods.
    }


    \section{Data, experiments and findings}
    \contentdescription{
        Data, experiments and findings (30-40\%):

        Describe the data you are working with for your project. What type of data is it? Where did it come from? How much data are you working with? Did you have to do any preprocessing, filtering, or other special treatment to use this data in your project?
        Describe and present the experiments that you performed and what is the reason for those experiments. Where applicable define evaluation metrics that you used. Discuss the results that you got.
    }


    \section{Challenges and Conclusions}
    \contentdescription{
        Challenges and Conclusions (5-15\%):
        Challenges you faced when reimplementing the paper and conducting the experiments. Were all details in the paper? Or did you have to look in the authors code or even contact them to find about some details? Was parts of the code quite hard to get them to work as intended? Did you have optimize and tune several hyperparameters? Which ones? Did the framework you used make the implementation difficult in some ways?

        Summarize your key results - what have you learned? What points do you think one should consider when using the approach of the paper you chose for your project? Suggest ideas for future extensions or new applications of your ideas.
    }


    \section{Ethical consideration, societal impact, potential alignment with UN SDGs}
    \contentdescription{
        Ethical consideration, societal impact, potential alignment with UN SDGs (5-10\%): Think and research! Are there any ethical considerations for the original paper, its problem or method, its way of conducting experiments? How about your task, your datasets, and the experiments you did? What societal impact can you imagine about the original paper and its contributions and results? How about your project report? How do you think this paper can push the UN SDG targets?
    }

    \begin{ack}
        This work is partially supported by the Wallenberg AI, Autonomous Systems and Software Program (WASP) funded by the Knut and Alice Wallenberg Foundation.
    \end{ack}

\end{document}

