\documentclass[12pt]{article}

\usepackage[preprint]{neurips_2021}

\usepackage[utf8]{inputenc} % allow UTF-8 input
\usepackage[T1]{fontenc}    % use 8-bit T1 fonts
\usepackage{hyperref}       % hyperlinks
\usepackage{url}            % simple URL typesetting
\usepackage{booktabs}       % professional-quality tables
\usepackage{amsfonts}       % blackboard math symbols
\usepackage{nicefrac}       % compact symbols for 1/2, etc.
\usepackage{microtype}      % microtypography
\usepackage{xcolor}

\usepackage{amsmath}

\usepackage{graphicx}

\graphicspath{ {./images/} {./images/ae/} {./images/bad/} {./images/curves/} {./images/latent/} {./images/raw/} }

\usepackage{amsthm}
\newtheorem{definition}{Definition}

\usepackage{float}

\newcommand{\contentdescription}[1]{}

\newcommand{\TODO}[2]{{#2}}

\title{Generating 3D Point Cloud Data with Generative Adversarial Networks and Autoencoders}

\author{
    Pavlo Melnyk$^*$ \\
    \texttt{pavlo.melnyk@liu.se} \\
    \And
    Julian Alfredo Mendez$^\star $ \\
    \texttt{julian.mendez@umu.se} \\
    \And
    Emanuel S\'{a}nchez Aimar$^*$ \\
    \texttt{emanuel.sanchez.aimar@liu.se} \\
    \\
    {\small $^*$Computer Vision Laboratory, Department of Electrical Engineering, Linköping University} \\
    {\small $^\star$ Department of Computing Science, Ume{\aa} University}
}

\begin{document}

    \maketitle

    \begin{abstract}
        \contentdescription{
            Abstract (5-10\%):
            Give an overview of what you have done in the project with the key results and findings of your work.
            Should be no more than 300 words.
        }

        Recognition and representation of 3D geometric data is a challenging and fruitful topic that has fast developed in recent years. An intuitive way to represent such type of data is as a set of 3D locations in a Euclidean coordinate frame, i.e., as a \textit{point cloud}. However, as for other types of data, generating 3D point clouds poses a unique set of challenges, which inspired us to consider this task.
        In this report,  we analyze and implement the raw and latent generative adversarial network (r- and l-GAN, respectively) methods proposed in the original work~\cite{pmlr-v80-achlioptas18a} to generate point clouds of 3D shapes of different categories using the ShapeNet\footnote{\url{https://shapenet.org}} data. The main feature of the latter is the use of a deep autoencoder (AE) network that is trained by minimizing the Chamfer pseudo-distance and has ``state-of-the-art reconstruction quality and generalization ability''.
        Importantly, apart from (subjective) qualitative assessment of the generated geometric data, we perform quantitative evaluation of the generative models by using the introduced measures of sample fidelity (i.e., \textit{minimum matching distance}) and diversity (i.e., \textit{coverage}) based on matchings between sets of point clouds.
        The original work implementation\footnote{\url{https://github.com/optas/latent_3d_points}} is in TensorFlow\footnote{\url{https://www.tensorflow.org}}, whereas we use PyTorch\footnote{\url{https://pytorch.org}} to build the models and conduct experiments.
        This report and our code and results are available at \url{https://github.com/julianmendez/jep}.
    \end{abstract}


    \section{Introduction}

    \contentdescription{
        Introduction (5-15\%):
        Describe the problem, the approach of the paper, the experiments, and the results.
        At the high-level talk about what you worked on in your project and why it is important.
        Then give an overview of your results.
    }

    Three-dimensional (3D) data are used in numerous domains, ranging from augmented and in virtual reality, e.g., in gaming, to medicine.
    An intuitive approach for representing 3D data is \emph{point cloud}, which is a set of 3D locations in a Euclidean frame.
    When considering neural network (NN) architectures for processing point clouds, one faces a unique set of challenges. Namely, the requirement of \textit{permutation invariance}, which we define in Section~\ref{methods}.
    In this context, generating point clouds is a particularly interesting task.
    Therefore, in our work, we use PyTorch and implement two approaches to generate 3D synthetic shapes exploiting generative adversarial networks (GANs) and and autoencoders (AEs), as proposed in~\cite{pmlr-v80-achlioptas18a}. Following the original work, we utilize the ShapeNet data for training the models. We present both qualitative and quantitative assessment of the results, as well as analysis of failure cases.


    \section{Related work}
    \contentdescription{
        Related Work (5-15\%):
        Discuss the published work related to your project paper, the types of experiments you do and the additional method that you have added to this work or you have compared this paper with (if any).
    }

    One of the first successful deep neural network approaches for performing tasks involving 3D point clouds is PointNet~\cite{arxiv:1612.00593},
    which is ``a unified architecture for applications ranging from object classification, part segmentation, to scene semantic parsing''.
    It is a type of neural network that takes point clouds as input directly and respects the permutation invariance of the points.

    Another method uses point clouds only as an intermediate or output representation~\cite{arxiv:1612.02808, arxiv:1612.00603}.
    In~\cite{pmlr-v80-achlioptas18a}, the proposed architecture is based on autoencoders (AEs)~\cite{doi:10.5555/65669.104451, arxiv:1312.6114},
    and generative adversarial networks (GANs)~\cite{NIPS2014_5ca3e9b1, arxiv:1511.06434, arxiv:1612.02136}.
    These techniques are used to generate samples from complex underlying distributions and are the main focus in our report.


    \section{Methods}
    \label{methods}
    \contentdescription{
        Methods (15-25\%):
        Describe the original paper's method to the extent that you would need to make your report and findings understandable.
        Otherwise, here you can describe other methods that you compare with or other methods that you apply on top of what you reimplemented.
        Here, you also try to justify any methodical modification or incremental changes that you have added to the original paper.
        It may be helpful to include figures, diagrams, or tables to describe your method or compare it with other methods.
    }

    To facilitate the reader's understanding of the original work, as well as the content of Sections~\ref{sec:data_experiments_findings}~and~\ref{sec:conclusions} of our report, we describe the main concepts in the form of definitions below.

    \begin{definition}
        \normalfont
        A \emph{point cloud} is a set of 3D locations defined by their $(x, y, z)$ coordinates in a Euclidean coordinate frame \cite{pmlr-v80-achlioptas18a}. Therefore, a point cloud can be represented as an $N \times 3$ matrix, where N is the number of points.
    \end{definition}

    In this paper, a point cloud usually represents a 3D shape surface.

    \begin{definition}
        \normalfont
        The \emph{Chamfer (pseudo)-distance (CD)}\footnote{
            A pseudo-distance $d$ allows that $d(x, y) = 0$ for different points $x$, $y$.} between point clouds $S_1$ and $S_2$ is defined as follows:

        \begin{equation}
            d_{CD}(S_{1}, S_{2}) =
            \sum_{x \in S_{1}} \min _{y \in S_{2}} || x - y||_{2}^{2} + \sum_{y \in S_{2}} \min_{x \in S_{1}} ||x - y||_{2}^{2}
            \label{equation:chamfer_distance}
        \end{equation}

    \end{definition}

    % The EMD and the CD are used to measure the sampling error of the results.

    In this work, the application of CD is two-fold. First, CD defines a reconstruction loss to train autoencoders presented in definition 4. Second, following the original paper, we implement high level metrics in definition 7 and 8, namely fidelity and coverage, in terms of CD.

    \begin{definition}
        \normalfont
        The \emph{autoencoder} architecture is a design to reconstruct its input, and is composed by an \emph{encoder} (E) and a \emph{decoder} (D). The pipeline is represented in Figure~\ref{figure:diagram_of_autoencoder}.
        % \[x \to E \to z \to D \to \hat{x}\]

        \begin{figure}[H]
            \centering
            \begin{tabular}{cc}
                \includegraphics[width = 50mm]{autoencoder}
            \end{tabular}
            \caption{Diagram of an autoencoder as described in~\cite{pmlr-v80-achlioptas18a}.}
            \label{figure:diagram_of_autoencoder}
        \end{figure}

        The encoder takes an input $x$ and produces a compressed version $z$, which is called the \emph{latent representation} of $x$. The decoder tries to reconstruct $x$ by using $z$ as input, and returns $\hat{x}$ as output.
    \end{definition}

    The basic architecture used in this paper is the generative adversarial networks (GANs).

    \begin{definition}
        \normalfont
        A \emph{generative adversarial network} (GAN) \cite{NIPS2014_5ca3e9b1} is an interplay between a \emph{generator} (G) and a \emph{discriminator} (D).
        In that interaction, the generator tries to synthesize samples that mimic real data by giving a randomly created sample through the generator function, while the discriminator needs to distinguish the synthesized examples from the real ones. This adversarial process is illustrated in Figure~\ref{figure:diagram_of_gan}. During training, we alternate the optimization of the losses presented in \eqref{equation:disc_loss} and \eqref{equation:gen_loss} for discriminator and generator, respectively.

        \begin{figure}[H]
            \centering
            \begin{tabular}{cc}
                \includegraphics[width = 50mm]{gan}
            \end{tabular}
            \caption{Diagram of a GAN as described in~\cite{pmlr-v80-achlioptas18a}.}
            \label{figure:diagram_of_gan}
        \end{figure}

        \begin{equation}
            \mathcal{L}_{dis} = \mathop{\mathbb{E}}_{x \sim p_x(x)}[log D(x)] + \mathop{\mathbb{E}}_{z \sim p_z(z)}[log (1 - D(G(z))]
            \label{equation:dis_loss}
        \end{equation}

        \begin{equation}
            \mathcal{L}_{gen} = \mathop{\mathbb{E}}_{z \sim p_z(z)}[log D(G(z))]
            \label{equation:gen_loss}
        \end{equation}

    \end{definition}

    \begin{definition}
        \normalfont
        \textit{Permutation invariance} of a point cloud implies that any reordering of the points yields a point cloud that represents the same shape. This complicates comparisons between two point sets which is needed to define a reconstruction loss, as noted in~\cite{pmlr-v80-achlioptas18a} and requires a permutation invariant learning of encoded features.

        We say that a model (e.g., a neural network classifier) is \emph{permutation invariant} when it does not assume any spatial relationships between the features in the input. E.g., if one permutes the points in the input point set and the performance of the model remains the same, then the model is permutation invariant.
        The generative models considered in this work are permutation invariant by construction since they contain 1-D convolutional layers with kernel size 1, allowing encoding the points \textit{independently}, and a symmetric (permutation invariant) max-pooling operation over the feature dimension.
        % ...
    \end{definition}

    \begin{definition}
        \normalfont
        Given two sets of point clouds $A$ and $B$, and given a distance, we say that \emph{coverage} is a measure for the fraction of the point clouds in $B$ that were matched to point clouds in $A$, such that each point cloud in $A$ is related to its the closest neighbor in $B$ using the given distance.
    \end{definition}

    \begin{definition}
        \normalfont
        We define the \emph{fidelity} of $A$ with respect to $B$ as the \emph{minimum matching distance} (MMD) of matching each point in $B$ to a point in $A$.
        The \emph{MMD} is computed as an average of the individual point-wise distances, using either the CD or EMD.
        They are called the \emph{MMD-CD} and \emph{MMD-EMD} respectively.
    \end{definition}

    Importantly, fidelity and coverage are meant to measure the global error of the computation and serve as a tool for quantitative assessment of the generated samples.

    % Let us see an intuitive example behind fidelity and the MMD.
    % If we sample chairs, we can look at the closest matching chair from the generated ones for a given metric.


    \section{Data, experiments and findings}
    \label{sec:data_experiments_findings}

    Following the original work \cite{pmlr-v80-achlioptas18a}, we use the ShapeNet dataset~\cite{arxiv:1512.03012} consisting of 16 classes of shapes
    %(\verb|"Airplane", "Bag", "Cap", "Car", "Chair", "Earphone", "Guitar", "Knife", "Lamp", "Laptop", "Motorbike", "Mug", "Pistol", "Rocket", "Skateboard", "Table"|)
    that are axis aligned and centered into the unit sphere, and provided as point clouds in the PyTorch Geometric\footnote{\url{https://pytorch-geometric.readthedocs.io/en/latest/}} package. Figure \ref{figure:samples_generated_with_autoencoders} (left hand-side column) presents training set examples for several categories.

    We use shapes of one class and uniformly sample 2048 points from each point cloud to train models. We use the default training + validation splits for training, and compute the metrics on the test split.

    We conduct two types of experiments: with raw GANs (\textit{r-GANs}) and latent GANs (\textit{l-GANs}). In the latter, we train deep AEs to learn a compressed representation of the point clouds. The architectures of all models are replicated from the original work (see Appendix in \cite{pmlr-v80-achlioptas18a} and the model summary outputs in the notebooks that we provide in the supplementary material). An important aspect of AE's encoder is the use of 1-D convolutions with $1 \times 1$ kernels, which ensures that learned transformations are applied point-wise \textit{independently}.

    \subsection{r-GAN experiments} As the name suggests, this type of GAN operates on the raw data, i.e., a $2048 \times 3$ matrix representing a 3D shape. We follow the specification in the original work, but stop the optimization after 200 epochs (instead of the suggested 2000) for the sake of computational efficiency. The generator takes as input a random vector of length $128$ sampled from a Gaussian distribution $\mathcal{N}(\mu=0,\sigma=0.2)$ and outputs a $2048 \times 3$ array.
    Generated samples of category \verb|Chair| are presented in Figure~\ref{figure:samples_generated_with_raw_gan}. Learning curves for the generator and discriminator losses (defined by \eqref{equation:gen_loss} and \eqref{equation:dis_loss}, respectively) are presented in Figure~\ref{figure:curves_for_latent_gan_and_raw_gan}.
    The quantitative results, i.e., the fidelity and coverage scores for the corresponding generated and test point cloud sets are shown in Table~\ref{table:results}.

    \subsection{l-GAN experiments} Instead of utilizing the raw data, the l-GAN takes as input a compressed representation of the original shape, which is obtained by passing the data through the encoder part of a pre-trained AE.

    We perform our l-GAN experiments in two steps:
    (1) For a given shape category, we train an autoencoder by minimizing the CD loss (Eq. ~\eqref{equation:chamfer_distance}) between the input and reconstructed point clouds. The output of the encoder, the bottleneck $z$, represents the compressed representation, i.e., \textit{encoding}.
    (2) Similarly to the r-GAN case, we sample a random vector of length $128$ from $\mathcal{N}(\mu=0,\sigma=0.2)$, but now we generate an \textit{encoding} representation of a point cloud (128-long vector) instead of a complete point cloud.

    We train each AE for 500 epochs, but perform only 200 epochs to optimize the parameters of the l-GANs, consistently with our choice for training the r-GANs.

    Original and AE-reconstructed samples are presented in Figure~\ref{figure:samples_generated_with_autoencoders}. Samples generated by l-GANs are shown in Figure~\ref{figure:samples_generated_with_latent_gan}. Illustrative failure cases are presented in Figure~\ref{figure:bad_samples_generated_with_latent_gan}. We present quantitative results in Table~\ref{table:results}.

    % \textbf{Stage 3}  This is used to sample and create using GANs
    % $C' = \sigma(z)$
    % $P = D(C')$

    % We used examples of a chair, a table, and an airplane.
    % Figure~\ref{figure:car_sampled_with_lGAN} show an example of a car sample generated with l-GAN.

    \begin{table}[H]
        \centering
        \begin{tabular}{lll}
            \toprule
            Model       & Fidelity $\downarrow$ & Coverage $\uparrow$ \\
            \midrule
            r-GAN Chair & 0.0022                & 14.35               \\
            l-GAN Chair & 0.0017                & 38.21               \\
            \midrule
            r-GAN Lamp  & 0.0196                & 17.48               \\
            l-GAN Lamp  & 0.0058                & 18.18               \\
            \midrule
            r-GAN Table & 0.023                 & 22.05               \\
            l-GAN Table & 0.002                 & 34.91               \\
            \midrule
            % r-GAN Airplane & 0.001                 & 12.02               \\
            % l-GAN Airplane & 0.0007                & 27.27               \\
            % \bottomrule
        \end{tabular}
        \caption{Fidelity and coverage comparison between r-GAN and l-GAN
        model on the test split of each class. A lower score is better for fidelity, whereas a higher score is better for coverage.}

        \label{table:results}
    \end{table}

    \subsection{Results}

    As can be seen in Figure~\ref{figure:samples_generated_with_raw_gan}, the quality of the resulting r-GAN generated samples of \verb|Chair| is reasonably good, considering the 10-fold shorter optimization we performed compared to the baseline. However, the r-GAN of \verb|Lamp| fails to generate reasonable samples. This is well reflected in Figure~\ref{figure:curves_for_latent_gan_and_raw_gan}. On one hand, the \verb|Chair| r-GAN converges nicely to an equilibrium (see Figure~\ref{figure:curves_for_latent_gan_and_raw_gan} (c)). In contrast, \verb|Lamp| r-GAN fails to converge (see Figure~\ref{figure:curves_for_latent_gan_and_raw_gan} (d)). A possible cause of this phenomenon is the amount of training data: we observe a strong correlation between the number of training samples (see Table \ref{table:statistics}) and the quality of shapes generated.

    \begin{table}[H]
        \centering
        \begin{tabular}{ll}
            \toprule
            Object       & Number \\
            \midrule
            \verb|Lamp|  & 1261   \\
            \verb|Table| & 4423   \\
            \verb|Chair| & 3054   \\
            \bottomrule
        \end{tabular}
        \caption{Number of samples in the training data.}
        \label{table:statistics}
    \end{table}

    \begin{figure}
        \centering
        \begin{tabular}{lllllll}
            \includegraphics[width = 30mm]{chair_raw_gen_1} &
            \includegraphics[width = 30mm]{chair_raw_gen_2} &
            \includegraphics[width = 30mm]{chair_raw_gen_3} \\
        \end{tabular}
        \caption{Samples of class Chair generated with r-GAN.}
        \label{figure:samples_generated_with_raw_gan}
    \end{figure}

    Figure~\ref{figure:samples_generated_with_latent_gan} illustrates the effectiveness of l-GAN. With enough training data, l-GAN is able to generate sharp and diverse samples, following different styles for \verb|Chair| and \verb|Table|. Additionally, in the low-data regime, \verb|Lamp| l-GAN shows lower diversity, an issue commonly known as \textit{mode collapse} in the literature. Despite the strong performance of l-GAN, we observe that it can sometimes generate noisy or ill-defined shapes, as observed in Figure~\ref{figure:bad_samples_generated_with_latent_gan}.

    Instrumental to l-GAN approach is the low-dimensional bottleneck representation from AE. In Figure~\ref{figure:samples_generated_with_latent_gan}, we observe that AE can reconstruct complex shapes with high level of detail.

    \begin{figure}
        \centering
        \begin{tabular}{lllllll}
            \includegraphics[width = 30mm]{chair_latent_gen_1} &
            \includegraphics[width = 30mm]{chair_latent_gen_2} &
            \includegraphics[width = 30mm]{chair_gen_4} &
            \includegraphics[width = 30mm]{chair_gen_5} \\
            \includegraphics[width = 30mm]{table_latent_gen_1} &
            \includegraphics[width = 30mm]{table_latent_gen_2} &
            \includegraphics[width = 30mm]{table_gen_5} &
            \includegraphics[width = 30mm]{table_latent_gen_4} \\
            \includegraphics[width = 30mm]{lamp_latent_gen_1} &
            \includegraphics[width = 30mm]{lamp_latent_gen_2} \\

        \end{tabular}
        \caption{Samples generated with l-GAN.}
        \label{figure:samples_generated_with_latent_gan}
    \end{figure}

    \begin{figure}
        \centering
        \begin{tabular}{lllllll}
            \includegraphics[width = 60mm]{chair_ae_1} &
            \includegraphics[width = 60mm]{chair_ae_2} \\
            \includegraphics[width = 60mm]{lamp_ae_1} &
            \includegraphics[width = 60mm]{lamp_ae_2} \\
            \includegraphics[width = 60mm]{lamp_ae_3} \\
            \includegraphics[width = 60mm]{table_ae_1} &
            \includegraphics[width = 60mm]{table_ae_2} \\
        \end{tabular}
        \caption{Reconstruction of shapes using AE. The leftmost image of each pair shows the ground truth shape, the rightmost the shape produced after encoding and decoding using a class-specific AE.}
        \label{figure:samples_generated_with_autoencoders}
    \end{figure}

    \begin{figure}
        \centering
        \begin{tabular}{lllllll}
            \includegraphics[width = 30mm]{chair_latent_gen_3_bad} &
            \includegraphics[width = 30mm]{lamp_latent_gen_3_bad} &
            \includegraphics[width = 30mm]{table_latent_gen_3_bad} \\
        \end{tabular}
        \caption{Failure cases generated with l-GAN. From left to right, generated images belong to Chair, Lamp and Tables class, respectively.}
        \label{figure:bad_samples_generated_with_latent_gan}
    \end{figure}


    \begin{figure}
        \centering
        \begin{tabular}{lllllll}
            \includegraphics[width = 30mm]{chair_l_gan_curves} &
            \includegraphics[width = 30mm]{table_l_gan_curves} &
            \includegraphics[width = 30mm]{chair_raw_gan_curves} &
            \includegraphics[width = 30mm]{lamp_raw_gan_curves} \\
        \end{tabular}
        \caption{Training loss as a function of training epochs. From left to right, (a) Chamfer loss for AE trained on Chair data, (b) Chamfer loss for AE trained on Table data, (c) GAN losses for r-GAN trained on Chair data, (d) GAN losses for l-GAN trained on Lamp data.}
        \label{figure:curves_for_latent_gan_and_raw_gan}
    \end{figure}


    \section{Challenges and conclusions}
    \label{sec:conclusions}
    \contentdescription{
        Challenges and Conclusions (5-15\%):
        Challenges you faced when reimplementing the paper and conducting the experiments.
        Were all details in the paper?
        Or did you have to look in the authors code or even contact them to find about some details?
        Was parts of the code quite hard to get them to work as intended?
        Did you have optimize and tune several hyperparameters?
        Which ones?
        Did the framework you used make the implementation difficult in some ways?

        Summarize your key results - what have you learned?
        What points do you think one should consider when using the approach of the paper you chose for your project?
        Suggest ideas for future extensions or new applications of your ideas.
    }

    We are able to successfully reproduce the results in~\cite{pmlr-v80-achlioptas18a}.

    Utilizing the latent space GANs (l-GANs), we can achieve better reconstructions, as discussed in the original work.
    By contrast, the raw GAN performance is considerably lower when not enough data is provided.

    For example, in some of our experiments we found that the lamp data were insufficient to train the raw GAN reasonably well.


    \section{Ethical consideration, societal impact, potential alignment with UN SDGs}
    \contentdescription{
        Ethical consideration, societal impact, potential alignment with UN SDGs (5-10\%):
        Think and research!
        Are there any ethical considerations for the original paper, its problem or method, its way of conducting experiments?
        How about your task, your datasets, and the experiments you did?
        What societal impact can you imagine about the original paper and its contributions and results?
        How about your project report?
        How do you think this paper can push the UN SDG targets?
    }

    The Sustainable Development Goals (SDGs), were adopted by the United Nations in 2015 as a universal call
    to end poverty and protect the planet\footnote{\url{https://sdgs.un.org/goals}}, intended to be achieved by the year 2030.
    % For the sake of completeness, we mention the goals:
    % \begin{enumerate}
    %    \item No Poverty,
    %    \item Zero Hunger,
    %    \item Good Health and Well-being,
    %    \item Quality Education,
    %    \item Gender Equality,
    %    \item Clean Water and Sanitation,
    %    \item Affordable and Clean Energy,
    %    \item Decent Work and Economic Growth,
    %    \item Industry, Innovation and Infrastructure,
    %    \item Reducing Inequality,
    %    \item Sustainable Cities and Communities,
    %    \item Responsible Consumption and Production,
    %    \item Climate Action,
    %    \item Life Below Water,
    %    \item Life On Land,
    %    \item Peace, Justice, and Strong Institutions,
    %    \item Partnerships for the Goals.
    %\end{enumerate}

    Use of GANs can impact indirectly in all of the goals.
    However, we have chosen three items, which we consider the most directly affected:
    \textbf{Decent Work and Economic Growth}, \textbf{Industry, Innovation and Infrastructure}, and
    \textbf{Sustainable Cities and Communities}.

    We base our position on the fact ethical uses of machine learning can have a positive impact in societal development.
    Technology can leave humans in overseeing positions, avoiding dangerous or repetitive tasks.
    In addition, automation of processes, like the ones presented in this paper, can lead to a more efficient use of resources.

    GANs can also be used to deceive systems.
    The physical adversarial attack is analyzed by~\cite{arxiv:1812.10217} where they discuss
    how a system can be used to confuse face recognition in authentication and objection detection in autonomous driving cars.

    Such systems should be deployed only if the ethical stakeholders accept them.

    Especially, for a specific task, the first questions is: ``Should be use an AI system for that?''.

    We went the extra mile about the ethical consequences of AI and found how a major company like Google faces this issue.
    The company explicitly stands against developing AI which principal goals are overall harm, weapons or tools to injure people, surveillance violating international norms, and contravention of principles in international law and human rights\footnote{\url{https://ai.google/principles/}}.

    We also considered that our application respects the 7 values, as mentioned for example in~\cite{easa:20210401.01}.
    They are: Human agency and oversight, Technical robustness and safety, Privacy and data governance, Transparency, Diversity, non-discrimination and fairness, Societal and environmental well-being, and Accountability.
    We also took into account these 3 principles: Accountability, Responsibility, and Transparency (ART)\cite{doi:10.1145/3278721.3278745}.

    \begin{ack}
        This work is partially supported by the Wallenberg AI, Autonomous Systems and Software Program (WASP) funded by the Knut and Alice Wallenberg Foundation.
    \end{ack}

    \bibliographystyle{plain}

    \bibliography{bibliography}

\end{document}

